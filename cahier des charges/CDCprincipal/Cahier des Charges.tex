\documentclass{article}
\usepackage[francais]{babel}
\usepackage[utf8]{inputenc}
\usepackage[T1]{fontenc}
\usepackage{graphicx}
\usepackage{fancyhdr}
\usepackage{eurosym}
\usepackage{color}
\usepackage{soul}

\pagestyle{fancyplain} \chead{}\lhead{\textit{Alpha-c}} \rhead{\emph{\textit{The Forust}}}



\begin{document}
\thispagestyle{empty}
\begin{center}
 \fontsize{42}{42}{\textbf{Cahier des charges \newline The forust}}
\end{center}

\begin{center}
 \fontsize{21}{21}{\textbf{Alpha-c }}
\end{center}

\vspace*{0.5cm}

\begin{center}
\includegraphics[scale=0.2]{/Users/pierrejacobe/Desktop/alphac.png}
\end{center}

\vspace*{0.5cm}

\fontsize{14}{14}
\begin{center}
{Jacobé
\textcolor{blue}{"Way hd"}
Pierre
\end{center}
\begin{center}
Boisson \textcolor{blue}{"zraulix"} Brice
\end{center}
\begin{center}
Guérin \textcolor{blue}{"Shmiti"} Jean
\end{center}




\newpage
\thispagestyle{empty}
\tableofcontents



\newpage
\fontsize{12}{12}
\pagenumbering{arabic}
\section{Introduction}

\paragraph{}
Un projet crée  par le groupe \textcolor{blue}{"alpha-c"}, ce fabuleux  projet  durera une durée de 6 mois.
\newline


\subsection{Origines du groupe}

\par
Nous nous sommes rencontrés à l'EPITA cette année, le feeling est tout de suite bien passé entre nous, en effet la passion du  gaming et des nouvelles technologies rapprochent facilement. Ce groupe permettera sans doute de souder  une  solide amitié entre nous.
\newline

\par
Il est vrai que les jeux vidéos peuvent prendre le dessus sur le temps de travail  d'un élève, mais lorsqu'il s'agit de faire un jeu cela devient tout de suite plus intéressant. Notre passion commune du  jeu  vidéo aura été fédératrice sur le sujet  abordé pour ce projet.
\newline

\par
En  revanche le choix du types du jeu vidéo que nous allions réalisé, c'est avéré plus difficile, en effet nous sommes tous passionés par des types de jeu différents et donc de  ce fait il y'avait des désaccord. Mais finalement le type jeu de survie aura su ravir tout le monde, en effet c'est un sujet  en vogue et qui nous pensons, pourra être formateur pour nous avec ses nombreuses perspectives. 
\newline

\par
C'est donc avec pleins d'ambitions et d'idées que nous allons attaquer ce  projet.



\newpage
\subsection{Les  membres}
\subsubsection{Pierre jacobé aka \textcolor{red}{"Wayhd"}}

\par
J'aime bien les choses classiques donc je vais me présenter simplement, tout d'abord je m'appel Pierre, j'ai 18 ans et je suis  en première années à  Epita  (strasbourg bien évidement). Comme beaucoup  d'entre nous dans cette école j'imagine, je suis passioné par l'informatique depuis tout petit. Une passion qui doit sans doutes venir de mon paternel. J'ai d'ailleurs commencé très tôt à jouer au jeux vidéos. Bon à l'époque ce n'était pas des jeux de survie, car souvent trop complexe ou trop violent pour un enfant. Je suis venu à tout ce qui est logitiel de montage vidéo, style adobe première ou encore pour la retouche d'image photoshop, puis et venu la modelisation avec cinema 4D, j'étais comme on l'appel un "Youtubeur". J'ai réellement découvert la programmation à Epita, c'est quelque chose que j'aime beaucoup et que j'ai hate de mettre en pratique dans le jeu.
\newline

\par 
Le jeu de survie ma tout de suite entrainé car grand adepte de ce style et en général des mondes ouvert, je me suis tout de suite dit que cela mettrait à rude épreuve nos talents de level designer et autres, mais c'est finalement ce qui est intéressant dans ce projet : partir de rien, ce former, s'améliorer dans plein de domaine différents comme le sound design, la programmation, comme dit plus haut le level design et bien d'autres. Je me sens prêt à me confronter à tout cela avec joie et à y sacrifier quelques nuit.
J'espère y apporter ma motivation et mon sérieux durant ce projet.
\newline





\subsubsection{Brice Boisson alias \textcolor{red}{"Zraulix"}}

\par

\newline

\subsubsection{Jean Guerin alias \textcolor{red}{"Shmiti"}}

\par 
Je suis très motivé par la réalisation de ce projet. Ayant peu programmer avant mon arrivée à EPITA, mais ayant à chaque fois adorer le faire. J’ai cette année, enfin l’occasion de pouvoir en apprendre plus sur la matière. 
\newline
\par 
Ce projet sera un réel moyen d’approfondir mes connaissances ainsi qu’en acquérir de nouvelles. De plus, la réalisation d’un jeu vidéo décuple ma motivation, j’ai toujours été passionné par les jeux vidéo et j’ai toujours désirer savoir comment en réaliser un, alors ce projet sera l’occasion de le découvrir ! 
\newline

\par 
Le sujet « jeu de survie » m’inspire beaucoup car c’est un format qui me plaît et je pense qu’il y a beaucoup de possibilités de développement, que ce soit au niveau de la carte ou du scénario par exemple. 
\newline











\newpage
\subsection{Le projet}
\subsubsection{La génèse du projet}


\par
Le projet que nous allons vous détaillez dans ce cahier des charges sera donc un jeu de survie à la première personne. le concept ? Un jeu télévisé sur une déserte dont le candidat, représentant un joueur, devra s'extirper d'une ile dans un temps impartit tout en y survivant dans son environnement hostile, le joueur devra se débrouiller pour sa survie, c'est à dire manger et se défendre contre les différents monstres à l'aide d'armes.
\newline

\par
C'est donc un projet en C\# qui a été choisi à l'aide de Unity, pour nous le choix entre c# et ocaml s'est fait rapidement, et c'est notre côté geek qui nous à rassembler autour  d'un jeu. Pourquoi la 3D plutôt que la 2D ? Certes la 2D est certainement moins complexe à gérer, mais finalement la 3D s'est vite avéré plus adapté à notre je laissant un champ des possibilité plus grand. 
Le choix de la première personne par rapport à la 3 ème personne s'est aussi  fait naturellement, effectivement dans  ce genre de jeu au côté assez dramatique la première personne aura tendance à donner un côté plus réaliste, plus "prenant".
\newline

\par
Nous nous sommes inspirés de jeu étant déjà des références dans le genre à l'exemple de The Forest ou The rust que nous expliquerons dans la prochaine section et qui notamment donné naissance à notre jolie nom  de jeu  "The Forust" dans le but de combiner le meilleur des deux mondes, sans aucune prétention.
Notre principale inspiration vient donc de ces deux jeux.

\paragraph{}
Finalement le principe d'un jeu de survie est assez simple,  une carte, un joueur, des ennemis et des outils  pour aider le joueur. 

Dans ce jeu le joueur sera entièrement libre, c'est ce que  l'on appel un "open world" (monde ouvert). Le joueur  est donc entièrement  livré  à lui même que cela soit pour ces mouvements et   ses choix, rien ne lui est imposé si ce n'est les contraintes proposées par  le monde.
\newline
\subsubsection{Présentation du jeu}
\par 
Le but du jeu ? \textbf{laissez nous  vous expliquez : } 
Jeu de survie, le joueur arrive sur une ile avec un couteau, (style Man vs Wild) premièrement, il doit d'abord faire de quoi survivre, comme par exemple trouver  de l'eau potable parmi les différents cours d'eau du jeu, de la nourriture sous forme de fruit au sol, pêcher ect...  Le joueur devra fabriquer des outils de bases et armes a l’aide de ressources à ramasser (il n'aura pas d’outils  comme hache, pioche dans le but de limiter les interactions pour récupérer les ressources...). Ce qui limite les builds possibles et donc moins complexe à faire. De plus il devra gérer sa nourriture ainsi que son sommeil et sa soif pour améliorer ses chances de survies. En effet ces différents éléments rechargerons respectivement leurs points de sommeil, de faim et de soif sous peine s'ils y arrivent à zero de voir la vie du joueur baisser et d'être affaiblit en cas de combat. En revanche, ces trois élément seront  l'unique et seul moyens de regagner de la vie. Le joueur ne sera pas tout seul durant cet aventure, en effet il sera confronter à différents monstres n'ayant qu'un seul but : le tuer, en revanche ils seront une source de nourritures importantes.
\newline
\par 
De plus un un système de construction simplifié sera mit en place, en effet le ramassage de certaines ressources présentes sur la carte donnerons l'accès à des constructions dont le joueur pourra se servir et qui aiderons le joueur à sa survie dans ce monde hostile, à l'exemple de cabanes, feu de camps, lits et torches.
\newline

\par 
Attention aux joueurs qui voudront faire de la résistance la nuit tombée, la difficulté de survivre dans ce monde hostile changera considérablement, alors un conseil mettez vous à l'abris. Les monstres commenceront alors à attaquer pendant la nuit, durant les premières nuit de votre aventure n'ayez crainte ils seront inactif, en revanche comme la vie n'est pas un long fleuve tranquil cela changera au cours du temps. Durant son sommeil le joueur ne devra jamais être serein, il ne sera jamais totalement à l'abris de monstres. 
\newline

\par 
Parlons des monstres, ils seront aux nombres de trois avec différentes caractéristiques, un premier grand, fort et imposant mais qui a du mal à percevoir le joueur, un deuxième petit, assez faible et chétif mais qui à une vue, une ouïe sur-développé et enfin un dernier étant un mix des deux.
\newline


 

\par
Passons des à présents à l'état de l'art, quand est-il pour les autres jeux du genre?
\newline

\subsubsection{Etat de l'art}

\par  
Le genre du jeu de survie n'est pas nouveau, mais nous pensons qu'il faut distinguer deux choses, dans le fond dès que dans un jeu un joueur doit survivre face à des ennemis c'est un jeu de survie non ? 

Nous  n'aborderons ici que les jeux de survies style "seul contre tous" (entendez par la environnement, ennemis).
\newline

\par 
Le tout premier du genre définis ci-dessus est un jeu se nommant "UnReal World" sortit en 1992, ce jeu avait des graphismes type ASCII, ce qui était possible par les ordinateurs de l'époque.
\newline
\vspace*{0.5cm}

\begin{center}
\includegraphics[scale=1]{/Users/pierrejacobe/Desktop/Unknown.jpeg}
\end{center}


\begin{center}
	\emph{Unknown World, 1992 (toujours en développement).}
\end{center}

\par 
Le principe du jeu était assez simple, survivre dans des conditions extrêmes en Finlande à l'age de Fer. Le joueur devait survivre le plus longtemps que possible face à des hordes d'ennemis et des conditions météorologiques difficiles. Le jeu qui a vraiment popularisé le genre est minecraft avec son mode survie si connu aujourd'hui. Le joueur part de rien sur une carte généré aléatoirement et doit lutter contre montres, faire attention à sa santé, sa faim. Il peut exploiter le monde à souhait pour créer armes,  construction et autre.
\newline

\par
Aujourd'hui les jeux qui dominent le genre sont \textbf{the Forest et Rust}  le premier est sortie en 2018. Son concept ? Un joueur arrive sur une ile après un crash d'avion, Le joueur doit survivre en créant un abri, des armes, et des outils utiles à la survie. L'île est peuplée de diverses créatures dont une tribu de cannibales mutants qui vivent dans des villages et dans des grottes souterraines. 
\newline
\par 
Dans Rust, le joueur arrive dans un entrepôt désaffecté  sans réellement savoir pourquoi, il doit alors tout faire pour sa survie  et survivre face au scientifiques aux hélicoptères et aux tanks qui gardent des zones et qui n’hésitent pas à tirer si un joueur s’approche trop d’eux.
\newline

\par 
Comme vous avez pu le constater ce genre n'est pas récent et les fondamentaux sont souvent les mêmes, un monde hostile et peu de moyens à dispositions du joueurs.
\newline


\newpage
\section{Découpage du projet}
\subsection{Les commandes et le gameplay}
\subsubsection{Les commandes}

\par
Les commandes de notre jeu seront assez classique et connu du genre.
\newline

\par
\textbf{Les déplacements :} Le personnage se déplacera grâce au clavier configurer à sa convenance (notre jeu sera tout fois pensé pour un déplacement à l'aide des touches "ZQSD" ou "WASD" pour les claviers anglophones, Z pour avancer, Q pour allez à gauche, D pour allez à droite, S pour reculez) et bien sûr la touche espace pour sauter.
\newline

\par
\textbf{Les combats :} Deux touches possibles, le clic gauche de la souris pour frapper et le clic droit pour focus la cible, mais attention les armes s'abimerons à force d'utilisation et pourrons se casser.
\newline

\par
\textbf{Le ramassage :} Comme dans beaucoup de jeu du genre le ramassage se fera exclusivement avec la touche E du clavier car proche des touches de direction.
\newline

\par
Nous ne voulons pas complexifier le gameplay, juste rendre cela le plus simple et le plus efficace que possible, changer les habitudes du genre n'aurait que peut d'intérêt.
\newline

\subsubsection{Le gameplay}

\par
Le gameplay du jeu ne tournera que autour du mode solo.
\newline

\par
\textbf{Le combat :} 
Cet aspect est primordial dans notre jeu et certainement le plus complexe. Laissez-nous vous expliquer : En fonction des ressources que le joueur aura collecter, il pourra se confectionner de nombreuses armes avec chacune des spécificité différentes, nous devrons donc développer un système de combat prenant en charge cette diversité. En effet, des armes lourdes mais plus meurtrière donc infligeant plus de dégâts au ennemis mais plus handicapante ou des armes plus légère moins efficace, mais donnant plus de liberté de déplacement au joueur tout cela devra être introduit dans Unity. l'arme aura donc une influence sur le gameplay et le joueur devra faire des choix. 
\newline



\par
 De manière réciproque, les ennemis présents sur la carte pourrons eux aussi attaqué le joueur infligeant plus ou moins de pv en fonction de leur attributs.
\newline

\par 
La difficulté du jeu impactera le nombre d'ennemis et leur IA.
\newline
\newline
\newline
\newline
\par
\textbf{L'interaction :} En effet, le joueur pourra et devra interagir avec les éléments qui l'entoure. Il pourra notamment trouver différents types de bois et des minéraux. Des combinaisons de ses éléments donnerons naissance à différents objets pour le joueur, armes ou constructions. Ces éléments seront présents un peu partout sur la carte de manière aléatoire à chaque partie. Le joueur devra les chercher au sol en se baladant sur la map et les ramasser lorsqu'il sera à proximité. Nous devrons rendre possible toutes ces interactions à l'aide de Script C\# implémenté dans Unity.
\newline
\par 
Lorsque le joueur tuera un monstre, il pourra ramasser de la nourriture, utile à sa survie. Il pourra aussi se ressourcer en eau au bord de certains points d'eau, attention à l'empoisonnement. Tout cela à l'aide de l'unique touche E. Ces éléments utile à la survie du joueur devront être mit en place à l'aide d'Unity et du C\#.
\newline

\par
\textbf{Santé :} La santé fera partit intégrante du jeu et le joueur devra particulièrement attention à cette dernière car elle pourra être source d'arrêt de la partie. Nous développerons donc un système de gestion de la santé qui s'actualisera en temps réel pour que le joueur garde toujours en vu sa santé.
\newline

\par 
Le joueur pourra régénérer sa santé à l'aide de trois moyen : la nourriture, l'eau et le sommeil.
\newline

\par
Le jeu se termine lorsque le joueur atteint la barre critique de 0 point vie, lorsqu'il a survécu durant le temps impartit ou qu'il trouve un moyen secret de quitter l'ile (chute c'est un secret). Nous vous faisons pas de dessin, le premier vous avez perdu et les deux derniers gagner bien sûr.

\subsection{Le moteur physique et l'IA}

\par 

\newline



\newpage
\subsection{Les graphismes}

\par
Un jeu est toujours plus intéressant avec des graphismes flattant la rétine même s'ils ne font pas tout, nous pensons qu'ils seront un point primordial de notre jeu.
\newline

\subsubsection{les animations en jeu}
\par 
Notre jeu comportera de nombreuses animations, que cela soit en combat contre les différents ennemis les "coups donnés par armes", comme dit plus haut les armes auront des caractéristiques différentes et donc directement des animations en combat différentes. Mais aussi le ramassage des ressources, le joueur qui se nourrit ou qui boit et enfin les animations pour le sommeil du joueur lorsqu'il se couchera quelque part. Nous réaliserons ces différentes animations avec l'outil animation de Unity et les script C\#. 
\newline



\subsubsection{Modélisation }

\par 
La modélisation passera par l'apprentissage de logiciels comme Blender ou encore Cinéma 4D mais aussi à l'aide d'asset Unity. Les assets d'Unity sont des composants comme ici des modèles 3D, déjà réalisés par d'autres personnes. Il est vrai que refaire un arbre pour son jeu n'a finalement que peu d'interêt. Nous récupérerons aussi les modèles 3D d'animaux et monstres déjà fait (Un ours est un ours non ?). En revanche pour donner un aspect singulier à notre jeu, les armes et constructions seront réalisés à l'aide des logiciels cités ci-dessus. Elles se voudront le plus réalistes que possibles. Des textures déjà faites seront aussi récupéré et d'autre pourrons être réalisées à l'aide du logiciel Photoshop en fonction de nos besoins.
\newline

\subsubsection{Aspect graphique général}

\par 
L'aspect graphique général se voudra réaliste au possible, nous apporterons un soin particulier à la colorimétrie, c'est à dire tout ce qui touche aux couleurs du jeu, dans le but de  renforcer le sentiment d'insécurité du jeu, les couleurs seront ainsi assez ternes. L'environnement s'apparentera à une forêt, cela permettant de renforcer le côté sombre du jeu. La carte sera parsemer de rivières et lac et elle aura un relief très prononcé, nous ferons au maximum pour que le joueur est l'impression de s'y perdre. L'environnement comprenant, terrain, forêt, rivière et lac devront être développé à l'aide des assets Unity et de l'outil terrain du logiciel. Un cycle jour/nuit sera implémenté. Un éco-système simplifié de la foret sera crée avec quelques insects.
\newline



\newpage
\subsection{Les sons}

\par
 Nous savons tous l'importance de l'ambiance dans un jeu et plus particulièrement dans un jeu de survie, cela témoigne de ce que les créateurs on voulu faire ressentir dans leur jeu.
\newline

\par
\textbf{Ambiance sonore :} Il faudra une ambiance sonore tout au long du jeu, afin de maintenir une certaine tension sur le joueur. Le but est de rendre le jeu plus immersif grâce aux sons. Nous ne mettrons pas de musique car pas forcement immersif dans un jeu de survie, simplement le bruit produit par les différents éléments du jeu devant renforcer le sentiment d'insécurité du joueur. Le joueur devra être vigilant à chaque détail du son pouvant, par exemple, lui indiqué la position d'un ennemis.
\newline

\par
\textbf{Armes :} Chaque armes aura un son différent en fonction des matériaux de l'armes ou encore pour témoigner de son poids ou encore sa quelconque solidité ou fragilité 
\newline

\par
\textbf{Ennemis :} Les ennemis seront des monstres, il est donc normal que chaque animal est sa propre "ambiance sonore".
\newline

\par
\textbf{Personnage :} Le son que le personnage fait est aussi très important, les sons de ses pas changerons en fonction du sol, le souffle du joueur changera en fonction de sa fatigue.
\newline

\subsection{L'interface}

\subsubsection{Le menu principal}

\par
Le menu principal est un élément essentiel car il est le premier écran après la vidéo d'introduction. Il sera donc composé de plusieurs boutons :
\newline

\par
\textbf{Solo :} Permet l'accés à une partie, une fois ce bouton cliqué on demandera au joueur de choisir une difficulté parmi trois \textbf{: facile, moyen et difficile}.
\newline

\par
\textbf{Options :} Il sera possible de faire le mapping des touches, changer la résolution et régler le volume du jeu.
\newline

\par
\textbf{Quitter :} Comme son nom l'indique quitter le jeu et revenir au bureau.
\newline

\subsubsection{En jeu}

\par

\textbf{Interface du jeu :} En jeu l'interface ce distinguera en deux partie, l'ATH autrement dit "affichage tête haute" en français, étant un jeu de survie elle se voudra assez minimaliste donc : un timer pour le temps restant, une barre de vie étant impacté par la faim et la soif, un accès à l'inventaire du joueur par la touche "Tab". 
\newline

\par
\textbf{Le menu pause :} Il sera accessible en appuyant sur la touche espace. Il permettra de quitter le jeu, de reprendre le jeu ainsi qu'un accès au menu option.
\newline





\newpage
\subsection{Site Web}

\par
Le site web va être notre moyen de communiquer sur notre projet et de lui donner de la visibilité. Ce site contiendra les différentes versions de notre ainsi que les modifications qui y sont apportés. Il y contiendra également nos différents rapports tel que le cahier des charges ou les soutenances.
\newline

\par
Les rapport disponible au format LaTeX et pdf seront disponible au téléchargement et bien évidemment les différentes versions du jeu.
\newline

\par
Nous ajouterons une page de présentation globale du projet, une page de crédit pour les différents membre du groupe. Cette page contiendra également une section remerciement pour tout ce qui aura pu nous aider pour ce projet
\newline

\par
Une page media sera également disponible. Elle permettra au visiteur d'avoir un aperçu plus visuel de l'avancement de notre jeu, à l'aide de screenshot détaillés et de vidéo "trailer".
\newline

\par
Nous nous engageons à régulièrement mettre à jour le site pour permettre au visiteur d'avoir un suivi en temps réel du jeu.
\newline



\newpage
\section{Répartition des tâches}
\subsection{Répartition entre les membres du groupe}

\par
Une bonne répartition des taches est un groupes efficace. Nous tenons à ce chaque membre puisse faire avant tout de la programmation mais aussi de toucher à tout. 
\newline

\par
Un chef de tache sera désigné, il sera un charge de sa bonne intégration dans unity mais aussi de la correlation de sa tache avec l'ensemble du projet. Mettre plusieurs personnes sur une même tache permet ainsi d'être plus efficace
\newline

\par
Légende : $\times$ : Doit effectuer cette tâche   - \textcolor{red}{ $\times$ }: Chef de tache 

\begin{center}
\begin{tabular}{|c|c|c|c|}
\hline
 & Pierre & Jean & Brice \\
\hline

Commandes & $\times$ & $\times$&   \\
\hline
Système de ramassage &  & $\times$& $\times$  \\
\hline
Combat& $\times$ & & $\times$  \\
\hline
 Physique et IA & $\times$ & & $\times$  \\
 
\hline
Graphismes & $\times$ & $\times$ & \\

\hline
Sounds design &  & $\times$ & $\times$ \\
\hline
Interface & & $\times$ & $\times$ \\
\hline
Site Web &$\times$ &$\times$ &$\times$ \\
\hline
\end{tabular}
\end{center}

\begin{center}
\bf{Fig. 1 : Répartition initiale entre les membres du groupe}
\end{center}



\newpage
\subsection{Avancement}

\par
L'avancement est noté dans le tableau avec des pourcentages.
\newline

\subsubsection{1ère soutenance}

\begin{center}
\begin{tabular}{|c|c|c|c|}
\hline
 & Pierre & Jean & Brice \\
\hline
Commandes & 30\% &30\%  & \\
\hline
Système de ramassage &  & 30 \% & 30\% \\
\hline
Combat & 10\% & & 10\% \\
\hline
 physique et IA & 5\% & & 5\%  \\
\hline
Graphismes & 20\% & 20\% & \\
\hline
Sounds Design&  & 10\% & 10\% \\
\hline
Interface & & 20\% & 20 \% \\
\hline
Site Web & 20\%& 20 \%&20 \% \\ 
\hline
\end{tabular}
\end{center}

\begin{center}
\bf{Fig. 2 : Répartition à la 1ère soutenance}
\end{center}

\subsubsection{2ème soutenance}

\begin{center}
\begin{tabular}{|c|c|c|c|}
\hline
 & Pierre & Jean & Brice \\
\hline
Commandes & 30\% &30\%  & \\
\hline
Système de ramassage &  & 30 \% & 30\% \\
\hline
Combat & 10\% & & 10\% \\
\hline
 physique et IA & 5\% & & 5\%  \\
\hline
Graphismes & 20\% & 20\% & \\
\hline
Sounds Design&  & 10\% & 10\% \\
\hline
Interface & & 20\% & 20 \% \\
\hline
Site Web & 20\%& 20 \%&20 \% \\ 
\hline
\end{tabular}
\end{center}

\begin{center}
\bf{Fig. 3 : Répartition à la 2ème soutenance}
\end{center}



\newpage
\subsubsection{3ème soutenance}
\begin{center}
\begin{tabular}{|c|c|c|c|}
\hline
 & Pierre & Jean & Brice \\
\hline
Commandes & 30\% &30\%  & \\
\hline
Système de ramassage &  & 30 \% & 30\% \\
\hline
Combat & 10\% & & 10\% \\
\hline
 physique et IA & 5\% & & 5\%  \\
\hline
Graphismes & 20\% & 20\% & \\
\hline
Sounds Design&  & 10\% & 10\% \\
\hline
Interface & & 20\% & 20 \% \\
\hline
Site Web & 20\%& 20 \%&20 \% \\ 
\hline
\end{tabular}
\end{center}

\begin{center}
\bf{Fig. 4 : Répartition à la 3ème soutenance}
\end{center}

\subsubsection{4ème soutenance - Soutenance Finale}

\begin{center}
\begin{tabular}{|c|c|c|c|}
\hline
 & Pierre & Jean & Brice \\
\hline
Commandes & 30\% &30\%  & \\
\hline
Système de ramassage &  & 30 \% & 30\% \\
\hline
Combat & 10\% & & 10\% \\
\hline
 physique et IA & 5\% & & 5\%  \\
\hline
Graphismes & 20\% & 20\% & \\
\hline
Sounds Design&  & 10\% & 10\% \\
\hline
Interface & & 20\% & 20 \% \\
\hline
Site Web & 20\%& 20 \%&20 \% \\ 
\hline
\end{tabular}
\end{center}\begin{center}
\bf{Fig. 5 : Répartition à la soutenance finale (Complet)}
\end{center}



\newpage
\section{Coût de production}

\par
Un projet implique des coût pour permettre son bon fonction. En effet, par exemple un site n'est malheureusement pas gratuit. C'est pour cette raison que nous allons lister dans le tableau ci-dessous les différents coût engendrés par ce projet.
\newline

\begin{center}
\begin{tabular}{|c|c|}
\hline
Un vps  & 3.50 \euro /mois \\
\hline
Un nom de domaine & 3.99 \euro \\
\hline
Sweat à l'éffigie de Alpha-c& 90 \euro \\
\hline
Clé USB Alpha-c  & 20 \euro \\
\hline
Un tuto Udemy pour apprendre Unity & 10 \euro \\
\hline
T-Shirts Alpha-c & 80 \euro \\
\hline

\end{tabular}
\end{center}

\begin{center}
\bf{Fig. 6 : Coût de production }
\end{center}

\vspace*{2cm}

\begin{center}
\includegraphics[scale =1] {/Users/pierrejacobe/Desktop/images.jpeg}
\end{center}



\newpage
\section{Conclusion}

\par
Nous espérons aboutir ce projet, qui nous pensons sera bénéfique pour notre futur métier d'ingénieur, en effet il va nous enseigner de nombreuses choses comme le succès mais aussi l'échec, l'autonomie mais aussi et surtout la persévérance. C'est pourquoi nous avons dès à présent hate de tout mettre en oeuvre pour que ce projet soit une réussite pour tous.
\vspace*{1cm}
\par 
\begin{center}
	\emph{\textcolor{red}{In Alpha-C we trust }}
\end{center}

\vspace*{1cm}

\par
\begin{center}
\includegraphics[scale=0.2]{/Users/pierrejacobe/Desktop/alphac.png}


\end{document}
























